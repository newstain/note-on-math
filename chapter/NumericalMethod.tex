%%%%%%%%%%%%%%%%%%%%%%%%%%%%%%%%%%%%%%%%%%%%%%%%%%%%%%%%%%%%%
\chapter{插值方法}
许多实际问题都需要用$y=f(x)$来表示其内在规律,但其中相当一部分是通过实验或观测得到的,而且只能给出其中的某一区间$[ab]$内的离散点的数据。因此,我们希望能根据这些给定的数据构建出一个既能反应函数$f(x)$的特性,又便于计算的简单函数$P(x)$,用$P(x)$来近似$f(x)$。我们一般选取一类较简单的函数(如代数多项式或分段代数多项式)来作为$P(x)$,并使$P(x)=f(x_i)$对$i=0,1,\cdots,n$成立,这样的$P(x)$就是我们所希望得到的插值多项式。

假设函数$y=f(x)$在区间$[a,b]$内有定义,并且已知在点$a \leqslant x_0 < x_1 < \cdots < x_n \leqslant b$上的值$y_0$、$y_1$、$\cdots$、$y_n$,若存在一个简单函数$P(x)$,使
\begin{equation}
    P(x_i) = y_i,	i=0,1,\cdots,n
\end{equation} 
成立,则称$P(x)$为$f(x)$的\textbf{插值函数},点$x_0$、$x_1$、$\cdots$、$x_n$称为\textbf{插值节点},包含插值节点的区间$[a,b]$称为\textbf{插值区间},求插值函数$P(x)$的方法称为\textbf{插值法}。若$P(x)$是次数不超过$n$的代数多项式,即可写成如下形式
\begin{equation}
	P(x)=a_0 + a_1 x + \cdots + a_n x^n
\end{equation} 
其中$a_i$为实数,成$P(x)$为\textbf{插值多项式},相应的插值法称为\textbf{多项式插值}。

从图像上看,插值法相当于求曲线$y=P(x)$,使其通过给定的$n+1$个点$(x_i, y_i)$,并用它近似已知曲线$y=f(x)$。
\section{拉格朗日插值}
为理解清楚拉格朗日插值,我们先讨论比较特殊的线性插值和抛物线插值方法,然后再讨论拉格朗日插值方法,最后分析其截断误差并给出一些例子以供理解。
\subsection{线性插值与抛物线插值}


\section{三次样条插值方法}
下面我们先给出三次样条(cubic)插值的定义:
\begin{definition}{三次样条插值}
	函数$S(x) \in C^2 [a, b]$,存在给定节点$a = x_0 < x_1 < \cdots < x_i < \cdots < x_n = b$,若$S(x)$在每个小区间$[x_i, x_{i+1}]$上表现为三次多项式,则称$S(x)$是节点$x_0, x_1, \cdots, x_i , \cdots, x_n$上的\textbf{三次样条曲线}。若存在节点$x_i$上给定函数值$y_i = f(x_i)(i = 0, 1, 2, \cdots, n)$,并成立
	\begin{equation}
	    S(x_i) = y_i , i = 0, 1, \cdots, n
	\end{equation} 
	则$S(x)$称为\textbf{三次样条插值曲线}。	
\end{definition}
基于上述定义,在区间$[x_i, x_{i+1}]$内的$S(x)$可写为三次项形式
\begin{equation}
	S(x) = c x^3 + d x^2 + e x + f	\label{eq:cubic}
\end{equation} 
由上式我们发现一个区间内共有$c, d, e, f$这4个待定系数,范围在$[a, b]$内的所有区间$[x_0, x_1]$、$[x_1, x_2]$、$\cdots$、$[x_{n-1}, x_{n}]$共有$n$个,故全空间范围内一共有$4n$个待定系数。考虑到$S(x)$在两个连续区间$[x_{i-1}, x_i]$和$[x_i, x_{i+1}]$间是连续光滑的,因此其二阶导要求连续,由此可得到连续性条件:
	\begin{equation*}
		S(x_{j-0}) = S(x_{j+0}), \quad S^{\prime}(x_{j-0}) = S^{\prime}(x_{j+0}),\quad S^{\prime\prime}(x_{j-0}) = S^{\prime\prime}(x_{j+0}), \quad (j = 1, 2, \cdots, n-1)
	\end{equation*} 
	其中$x_{j-0}\in [x_{j-1}, x_j], x_{j+0} \in [x_{j}, x_{j+1}]$。从连续性条件我们可以得到$3n-3$个约束条件,再加上式\eqref{eq:cubic},我们可以得到$(3n-3) + (n + 1)= 4n - 2$个约束条件,但是一共有$4n$个待定系数,因此我们还需要补充2个条件才能求解出$S(x)$。这两个条件我们可以对$[a, b]$的两个端点$x_0 = a$和$x_n = b$分别施加一个条件(即\textbf{边界条件})获得,一般有以下3种取法:
	\begin{enumerate}
		\item 已知两端的一阶导数值
			\begin{equation}
				S^{\prime}(x_0) = f^\prime_0, \quad S^{\prime}(x_n) = f^\prime_n	\label{eq:first_boundary}
			\end{equation} 
		\item 已知两端的二阶导数值
			\begin{equation}
				S^{\prime\prime}(x_0) = f^{\prime\prime}_0, \quad S^{\prime\prime}(x_n) = f^{\prime\prime}_n	\label{eq:second_boundary}
			\end{equation} 
		我们把如下的特殊情况称为\textbf{自然边界条件}:
		\begin{equation}
			S^{\prime\prime}(x_0) = S^{\prime\prime}(x_n) = 0	\label{eq:natural_boundary}
		\end{equation} 
		\item 若函数$f(x)$是以$x_n - x_0$为周期的周期函数,则要求$S(x)$也是周期函数,此时边界条件要满足:
			\begin{equation}
				S(x_0 + 0 ) = S(x_n - 0), \quad S^{\prime}(x_0 + 0 ) = S^{\prime}(x_n - 0), \quad S^{\prime\prime} (x_0 + 0) = S^{\prime\prime}(x_n - 0)	\label{third_boundary}
			\end{equation} 
			此时式\eqref{eq:cubic}中满足$y_0 = y_n$,这样确定的样条函数$S(x)$称为\textbf{周期样条函数}。
	\end{enumerate}
	\subsection{三次样条曲线的构建}
	下面用插值函数$S(x)$的二阶导数值$S^{\prime\prime}$来表示$S(x)$。三次样条插值要求$S(x)$在区间$[x_i, x_{i+1}]$表现为三次函数形式:
	\begin{equation*}
		S(x) = Ax^3 + Bx^2 + Cx + D
	\end{equation*} 
	由上式,我们知道$S(x)$的二阶导$S^{\prime\prime}$表现为一次函数形式,也就是其二阶导在$[x_i, x_{i+1}]$上表现为线性函数形式。记$h_i = x_{i+1} - x_{i}$,我们在$[x_i, x_{i+1}]$将$S^{\prime\prime}$写为如下形式:
	\begin{equation}
		S^{\prime\prime} = M_{i+1} \frac{x - x_i}{h_i} + M_{i} \frac{x_{i+1} - x}{h_i}
	\end{equation} 
	对上式求两次积分
	\begin{equation*}
		\begin{aligned}
			S(x) =& \int\int S^{\prime\prime}(x) dx^2	\\
				 =&	\int\int M_{i+1} \frac{x - x_i}{h_i} + M_{i} \frac{x_{i+1} - x}{h_i} dx^2	\\
				 =&	\int M_{i+1} \frac{(x - x_i)^2}{2h_i} -  M_{i} \frac{(x_{i+1} - x)^2}{2h_i} + E dx	\\
				 =&	M_{i+1} \frac{(x - x_i)^3}{6h_i} +  M_{i} \frac{(x_{i+1} - x)^3}{6h_i} + Ex + F 
		\end{aligned}
	\end{equation*} 
	其中$E, F$为积分常数。考虑到$S(x_i) = y_i, S(x_{i+1}) = y_{i+1}$,我们有
	\begin{equation*}
		\left\{
			\begin{aligned}
				y_i =& M_{i+1} \frac{(x_{i} - x_i)^3}{6h_i} +  M_{i} \frac{(x_{i+1} - x_i)^3}{6h_i} + Ex_i + F  \\
					=& M_{i} \frac{(x_{i+1} - x_i)^3}{6h_i} + Ex_i + F = M_{i} \frac{h_i^2}{6} + Ex_i + F   \\ 
				y_{i+1} =& M_{i+1}\frac{(x_{i+1}-x_i)^3}{6h_i}+M_{i}\frac{(x_{i+1}-x_{i+1})^3}{6h_i}+Ex_{i+1}+F  \\
						=& M_{i+1} \frac{(x_{i+1} - x_i)^3}{6h_i} + Ex_{i+1} + F = M_{i+1} \frac{h_i^2}{6} + Ex_{i+1} + F   
			\end{aligned}
		\right.
	\end{equation*} 
	对上式做一下变形,得到
	\begin{equation*}
	    \left\{ 
			\begin{aligned}
				y_i - M_{i} \frac{h_i^2}{6} =&  Ex_i + F   \\ 
				y_{i+1} - M_{i+1} \frac{h_i^2}{6} = & Ex_{i+1} + F   
			\end{aligned}
		\right.
	\end{equation*} 
	将其作为线性方程的解,并利用对应的线性插值基函数,可得到最后的$S_{i}(x)$为:
	\begin{equation}
		\begin{aligned}
			S_{i}(x)=& M_i \frac{\left(x_{i+1}-x\right)^3}{6 h_i}+M_{i+1} \frac{\left(x-x_i\right)^3}{6 h_i}+\left(y_i-\frac{M_i h_i^2}{6}\right) \frac{x_{i+1}-x}{h_i} \\
				 &+\left(y_{i+1}-\frac{M_{i+1} h_i^2}{6}\right) \frac{x-x_i}{h_i}, \quad i=0,1, \cdots, n-1 .
		\end{aligned}
		\label{eq:unkown_Sx}
	\end{equation} 
	现在我们的问题就变成了求解式\eqref{eq:unkown_Sx},而其中的未知数为$M_i(i = 0, 1, \cdots, n)$。为确定未知数$M_i$,我们需要用到一阶导的连续性,为此,我们对$S_{i}(x)$求一次导得
	\begin{equation}
		S_{i}^{\prime}(x)=-M_i \frac{\left(x_{i+1}-x\right)^2}{2 h_i}+M_{i+1} \frac{\left(x-x_i\right)^2}{2 h_i}+\frac{y_{i+1}-y_i}{h_i}-\frac{M_{i+1}-M_i}{6} h_i	\label{eq:first_order_Sx}
	\end{equation}
	这样,在区间$[x_i, x_{i+1}]$内,我们得到
	\begin{equation*}
		S_{i}^{\prime}(x_i + 0) = - \frac{h_i}{2}M_i + \frac{y_{i+1} - y_i}{h_i} - \frac{M_{i+1} - M_{i}}{6}h_i = - \frac{h_i}{3}M_i - \frac{h_i}{6}M_{i+1} + \frac{y_{i+1} - y_i}{h_i}
	\end{equation*} 
	在区间$[x_{i-1}, x_i]$内,也可以类似的求得
	\begin{equation*}
		S^{\prime}(x_i - 0) = \frac{h_{i-1}}{2}M_{i} + \frac{y_{i} - y_{i-1}}{h_{i-1}} - \frac{M_{i} - M_{i-1}}{6}h_{i-1} = \frac{h_{i-1}}{3}M_i + \frac{h_{i-1}}{6}M_{i-1} + \frac{y_{i} - y_{i-1}}{h_{i-1}}
	\end{equation*} 
	再利用$S^{\prime}(x_i + 0) =S^{\prime}(x_i - 0)$,我们有
	\begin{equation*}
			- \frac{h_i}{3}M_i - \frac{h_i}{6}M_{i+1} + \frac{y_{i+1} - y_i}{h_i} = \frac{h_{i-1}}{3}M_i + \frac{h_{i-1}}{6}M_{i-1} + \frac{y_{i} - y_{i-1}}{h_{i-1}}	\\
	\end{equation*}
	由上式,我们可以直接得到对应的关系式为
	\begin{equation}
		\frac{h_{i-1}}{h_i + h_{i-1}}M_{i-1} + 2 M_i + \frac{h_{i}}{h_i + h_{i-1}}M_{i} = \frac{6}{h_i + h_{i - 1}}\left(\frac{y_{i+1} - y_i}{h_i} - \frac{y_i - y_{i - 1}}{h_{i - 1}} \right)	\label{eq:cubic_linear}
	\end{equation} 
	注意上式并不包含区域$[a, b]$上的端点,因此我们需要用边界条件来补全约束条件。
	\begin{enumerate}
		\item 第一种边界条件,满足式\eqref{eq:first_boundary},我们从式\eqref{eq:cubic_linear}中导出两个额外的方程。对于$x_0, x_n$,我们取一次导$S^{\prime}(0+0)$和$S^{\prime}(n-0)$,由式\eqref{eq:first_order_Sx},我们有
			\begin{equation}
				\begin{aligned}
					S^{\prime}(x_0+0) = f^{\prime}(x_0) =& - \frac{h_0}{3}M_0 - \frac{h_0}{6}M_{1} + \frac{y_{1} - y_0}{h_0}	\\
					\Rightarrow 
					2M_0 + M_1 =& \frac{6}{h_0} \left( \frac{y_1 - y_0}{h_0} - f^{\prime}(x_0) \right)	\\
					S^{\prime}(x_n-0) = f^{\prime}(x_n) =& \frac{h_{n-1}}{3}M_n - \frac{h_{n-1}}{6}M_{n-1} + \frac{y_{n} - y_{n-1}}{h_{n-1}}	\\
					\Rightarrow 
					M_{n-1} + 2M_n =& \frac{6}{h_{n-1}} \left( f^{\prime}(x_{n}) - \frac{y_{n} - y_{n-1}}{h_{n-1}} \right)	\\
				\end{aligned}
				\label{eq:first_boundary_condition}
			\end{equation} 
			这样,根据式\eqref{eq:cubic_linear}和\eqref{eq:first_boundary_condition},我们可以得到对应的矩阵形式如下:
			\begin{equation}
			    \begin{pmatrix}
					2						&	1	&	~						&	~	&	~\\
					\frac{h_0}{h_1 + h_{0}}	&	2	&	\frac{h_1}{h_1 + h_0}	&	~	&	~\\
					\ddots	&	\ddots		&	\ddots	&	\ddots	&	~\\
					~		&	~			&	\frac{h_{n-2}}{h_{n-1}+h_{n-2}}	&	2	&	\frac{h_{n-1}}{h_{n-2} + h_{n-1}}\\
					~		&	~			&	~	&	1	&	2\\
			    \end{pmatrix}
				\begin{pmatrix}
					M_0		\\
					M_1		\\	
					\vdots	\\
					M_{n-1}	\\
					M_{n}	
				\end{pmatrix}
				=
				\begin{pmatrix}
					\frac{6}{h_0}\left(\frac{y_1-y_0}{h_0} - f^{\prime}(x_0)\right)	\\
					\frac{6}{h_1 + h_0}\left(\frac{y_2-y_1}{h_1} - \frac{y_1 - y_0}{h_{0}}\right)	\\
					\vdots	\\
					\frac{6}{h_{n-1} + h_{n-2}}\left(\frac{y_n-y_{n-1}}{h_{n-1}} - \frac{y_{n-1} - y_{n-2}}{h_{n-2}}\right) \\
					\frac{6}{h_{n-1}}\left(f^{\prime}(x_n) - \frac{y_{n}-y_{n-1}}{h_{n-1}} \right)
				\end{pmatrix}
			\end{equation}
			由于$y_0, y_1, \cdots, y_n$和$h_0, h_1, \cdots, h_{n-1}$已知,通过求解上述线性方程组,我们可以得到对应的二阶导$M_0, M_1, \cdots, M_n$,将其带入式\eqref{eq:unkown_Sx}便可得到最后的插值多项式表示。
		\item 对于第二种边界条件,由于端点的连续性,我们直接有
			\begin{equation}
				M_0 = f^{\prime\prime}(x_0)	\quad M_n = f^{\prime\prime}(x_n)
			\end{equation} 
			这样可以得到相应的矩阵为
			\begin{equation}
			    \begin{pmatrix}
					1						&	0	&	~						&	~	&	~\\
					\frac{h_1}{h_1 + h_{0}}	&	2	&	\frac{h_0}{h_1 + h_0}	&	~	&	~\\
					\ddots	&	\ddots		&	\ddots	&	~	&	~\\
					~		&	~			&	\frac{h_{n-1}}{h_{n-1}+h_{n-2}}	&	2	&	\frac{h_{n-2}}{h_{n-2} + h_{n-1}}\\
					~		&	~			&	~	&	0	&	1\\
			    \end{pmatrix}
				\begin{pmatrix}
					M_0		\\
					M_1		\\	
					\vdots	\\
					M_{n-1}	\\
					M_{n}	
				\end{pmatrix}
				=
				\begin{pmatrix}
					f^{\prime\prime}(x_0)	\\
					\frac{6}{h_1 + h_0}\left(\frac{y_2-y_1}{h_1} - \frac{y_1 - y_0}{h_{0}}\right)	\\
					\vdots	\\
					\frac{6}{h_{n-1} + h_{n-2}}\left(\frac{y_n-y_{n-1}}{h_1} - \frac{y_{n-1} - y_{n-1}}{h_{n-2}}\right) \\
					f^{\prime\prime}(x_n)
				\end{pmatrix}
			\end{equation} 
	\end{enumerate}
	
%%%%%%%%%%%%%%%%%%%%%%%%%%%%%%%%%%%%%%%%%%%%%%%%%%%%%%%%%%%%%
\chapter{数值积分}

\section{一维数值积分}
\subsection{辛普森积分及复合辛普森积分}
\paragraph*{辛普森积分} 区间$[a, b]$有函数$f(x)$,求函数在此区间内的积分,可定义此区间的中点$\dfrac{a+b}{2}$及对应的函数值$f(\dfrac{a+b}{2})$,也就是对区间$[a, b]$进行二等分,这样,对应的数值积分为
\begin{equation}
	I = \int_{a}^{b} f(x) \, dx \approx \frac{h}{6}\left[ f(a) + f(\frac{a+b}{2}) + f(b) \right]	\label{eq:simpson-integer}
\end{equation} 

\paragraph*{复合辛普森积分} 区间$[a, b]$内有函数$f(x)$,现将区间$n$等分,在区间$[x_{k}, x_{k+1}](k = 0, 1, 2, \cdots, n-1)$内应用式\eqref{eq:simpson-integer},并对每个小区间进行累加,这样就可以得到整个区间内的积分。设$x_{k+1/2} = \dfrac{x_k + x_{k+1}}{2}$,这样,$[a, b]$区间内对$f(x)$的积分为
\begin{equation}
    \begin{aligned}
		I =& \int_{a}^{b} f(x) \, dx = \sum_{i = 0}^{n - 1}\int_{x_i}^{x_{i+1}} f(x) \, dx	\\
		=& \frac{h}{6}\sum_{i=1}^{n-1} \left[ f(x_i) + 4f(x_{k+1/2}) + f(x_{k+1})  \right] + R_n(f)
    \end{aligned}
\end{equation} 
取近似
\begin{equation}
    \begin{aligned}
    	I \approx S_n = \frac{h}{6}\sum_{i=1}^{n-1} \left[ f(x_i) + 4f(x_{k+1/2}) + f(x_{k+1})  \right]
    \end{aligned}
\end{equation} 
上式称为\uline{复合辛普森积分}。余项为
\begin{equation}
	R_n(f) = I - S_n = -\frac{h}{180} \left(\frac{h}{2} \right) \sum_{i = 0}^{n-1}f^{(4)}(\eta_i), \quad \eta_i \in (x_i, x_{i+1})
\end{equation} 
其中,$h = x_{i+1} - x_{k}$,误差阶为$h^4$。

\begin{example}
	\textbf{c++实现复合辛普森积分:} 考虑一些数据点,
	
\end{example}

\subsection{Cubic积分}
在完成cubic插值后,我们可根据方程\eqref{eq:unkown_Sx}求出原函数的近似,区间$[x_i, x_{i+1}]$区间内有函数$f(x)$,此时,我们有
\begin{equation}
	\begin{aligned}
		I_i =& \int_{x_i}^{x_{i+1}} f(x) \, dx \approx \int_{x_i}^{x_{i+1}} S(x) \, dx \\
		=& \int_{x_i}^{x_{i+1}} \, dx \left[ M_i \frac{\left(x_{i+1}-x\right)^3}{6 h_i}+M_{i+1} \frac{\left(x-x_i\right)^3}{6 h_i}+\left(y_i-\frac{M_i h_i^2}{6}\right) \frac{x_{i+1}-x}{h_i} \right. \\
		 &\left. + \left(y_{i+1}-\frac{M_{i+1} h_i^2}{6}\right) \frac{x-x_i}{h_i} \right]	\\
		=& \left[ -\frac{M_i}{24h_i} \left(x_{i+1}-x\right)^4 + \frac{M_{i+1}}{24h_i}\left(x-x_i\right)^4 \right. \\
		 &\left. -\left(y_i-\frac{M_i h_i^2}{6}\right) \frac{(x_{i+1}-x)^2}{2h_i} + \left(y_{i+1}-\frac{M_{i+1} h_i^2}{6}\right) \frac{(x-x_i)^2}{2h_i} + \text{Const} \right]_{x_i}^{x_{i+1}}	\\
		=& \frac{y_i + y_{i+1}}{2} h_i - \frac{M_{i} + M_{i+1}}{24} h_i^3
	\end{aligned}
\end{equation} 
其中,$h_i = x_{i+1}-x_{i}$。区间$[a,\ b]$存在函数$f(x)$,将此区间进行$n$等分,每个小区间
为$a = x_0 < x_1 < x_2 < \cdots < x_{n-1} < x_n = b $,求函数在此区域内的积分,如下
\begin{equation}
	I = \sum_{i=0}^{n-1} I_{i} \approx \sum_{i=0}^{n-1} \frac{y_i + y_{i+1}}{2} h_i - \frac{M_{i} + M_{i+1}}{24} h_i^3
\end{equation} 
当为等距格点时,即$h = h_0 = h_1 =\cdots = h_{n-1}$时,上式化为
\begin{equation}
	I\approx\frac{y_0+y_n+2\sum_{i=1}^{n-1}y_i}{2}h-\frac{M_0+M_n+2\sum_{i=1}^{n-1}M_i}{24}h^3
    \label{eq:cubic-int-equidistant}
\end{equation} 

\begin{example}{Cubic一维程序积分实现 —— 等间距格点}
    \begin{lstlisting}
        #include <iostream>
        #include <cmath>
        #include <Eigen/Sparse>
        #include <Eigen/SparseLU>

        double CubicInt(Eigen::VectorXd xvar, Eigen::VectorXd yvar, double xstep){
            // 获取二阶导数
            Eigen::VectorXd ypprime = getSectodDerivative(xvar, yvar, xstep);

            int dim = xvar.rows();  // 获取离散坐标的数量
            
            double sumyvar = 0;
            double sumypprime = 0;
            
            sumyvar = yvar[0] + y[dim - 1];             // 累加$y_0$和$y_n$
            sumypprime = ypprime[0] + ypprime[dim - 1]; // 累加$M_0$和$M_n$
            for(int i = 1; i < dim - 1; i++){
                sumyvar += 2. * yvar(i);                // 累加 2 * y_i
                sumypprime += 2. * ypprime(i);          // 累加 2 * M_i
            }

            double intvalue;
            int value = sumyvar * xstep * 0.5 - sumypprime * pow(xstep, 3) / 24.;
        }
    \end{lstlisting}
\end{example}

\section{二维平面积分}
对二维笛卡尔坐标系$x-y$平面的积分,可先对$x$坐标积分,在对$y$坐标积分,这样可得到原函数在
二维平面上的积分。
\subsection{Cubic二维平面积分}
对于等间距的点,我们直接用式\eqref{eq:cubic-int-equidistant}来分别对$x$方向和$y$方向进行
积分







%%%%%%%%%%%%%%%%%%%%%%%%%%%%%%%%%%%%%%%%%%%%%%%%%%%%%%%%%%%%%
\chapter{数值微分}
\section{五点数值微分公式}
\subsection{二阶导微分公式}
设$f(x)$为定义在区间$[a,b]$上的函数,给定$f(x)$在等距点$a \leqslant x_0 < x_1 < x_2 < x_3
< x_4 \leqslant b$,节点上函数值为$f(x_k)$,$(k=0, 1, 2, 3, 4)$,且满足$x_{k+1}-x_k=h$。
在$[a,b]$上做$f(x)$的4次Lagrange插值函数,并将$x=x_0+th, t \in [0, 4], x_k = x_0+kh$带入,
将方程两端对$t$求二次导,再分别把$t=0, 1, 2, 3, 4$带入,可得到$x_k$节点二阶导数的5点数值
微分公式,如下:
\begin{equation}
	\begin{aligned}
		f^{\prime\prime}(x_0) =& \frac{1}{12h^2}[35f(x_0) - 104f(x_1) + 114f(x_2) - 56f(x_3) + 11f(x_4)] \\
		f^{\prime\prime}(x_1) =& \frac{1}{12h^2}[11f(x_0) -  20f(x_1) +   6f(x_2) +  4f(x_3) -   f(x_4)] \\
		f^{\prime\prime}(x_2) =& \frac{1}{12h^2}[ -f(x_0) +  16f(x_1) -  30f(x_2) + 16f(x_3) -   f(x_4)] \\
		f^{\prime\prime}(x_3) =& \frac{1}{12h^2}[ -f(x_0) +   4f(x_1) +   6f(x_2) - 20f(x_3) + 11f(x_4)] \\
		f^{\prime\prime}(x_4) =& \frac{1}{12h^2}[11f(x_0) -  56f(x_1) + 114f(x_2) -104f(x_3) + 35f(x_4)]
	\end{aligned}
\end{equation} 




